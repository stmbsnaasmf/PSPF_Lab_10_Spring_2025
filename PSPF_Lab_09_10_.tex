\documentclass[12pt]{article}
\usepackage{graphicx}			    % Use this package to include images %Path relative to the main .tex file 
\graphicspath{ {./Images/} }
\usepackage{amsmath}			    % A library of many standard math expressions
\usepackage{mathtools}              % For Aboxed{} (https://tex.stackexchange.com/questions/346577/boxed-and-align)
% \usepackage[margin=1in]{geometry} % Sets 1in margins. 
\usepackage{fancyhdr}			    % Creates headers and footers
\usepackage{enumerate}              % These two packages give custom labels to a list
\usepackage[shortlabels]{enumitem}
\usepackage{hyperref}               % https://www.overleaf.com/learn/latex/Hyperlinks
\usepackage{xcolor}
\usepackage[svgnames]{xcolor}
\usepackage{float}
\usepackage{cmupint}                % For upright integrals. https://tex.stackexchange.com/questions/503527/how-to-write-upright-integrals-with-automatic-sizing
\usepackage{tikz}
\usetikzlibrary{trees}
\usepackage{titling}
\usepackage{minted}                 % For code blocks
\usemintedstyle{monokai}            % For code blocks
\definecolor{bg}{HTML}{282828}      % For code blocks, from https://github.com/kevinsawicki/monokai
\usepackage{nameref}
\usepackage{caption}                % To use \caption*{} to show only the caption text without "Table 1: <text>"
% \usepackage{mathtools, tccomicsans}
% \usepackage{comicsans}
% \usepackage[main,largesymbols]{tccomicsans} % https://www.reddit.com/r/LaTeX/comments/1l5no5d/comment/mwm64ze/
\renewcommand*\contentsname{Summary}
% \renewcommand{\contentsname}{\centering \normalfont\normalsize Contents}
\renewcommand{\contentsname}{\centering \bfseries\Large Contents}
% \renewcommand{\cftaftertoctitle}{\hfill}

\hypersetup
{
    colorlinks=true,
    linkcolor=blue,
    filecolor=magenta,      
    urlcolor=cyan,
    %pdftitle={Overleaf Example},
    pdfpagemode=FullScreen,
}

% \title{OOP Lab Manual 05}
% \author{STM}
% \date{December 2024}

\begin{document}

\begin{titlepage}
    \centering

    \vspace*{-8em}
    \includegraphics[width=0.5\textwidth]{Bismillah.png}%\\[2cm]
    \vspace*{5em}

    
    \vspace*{1cm}

     \includegraphics[width=0.5\textwidth]{GU Tech 1685x1330.png}\\[2cm]

    \MakeUppercase{\Huge \textbf{GU TECH}}\\[1.5ex]
    
    \vspace*{1cm}
    
    \Huge Problem Solving \& Programming Fundamentals \\[1.5ex]
    \LARGE Lab 09 + 10 \\[2cm]

    % {\Large STM} \\ [2cm]

    {\Large \today}\\[1cm]
    
\end{titlepage}

\newpage

% \vspace*{4cm}
% \begin{center}
%     \Huge \textbf{Outline}
% \end{center}

% \begin{itemize}
%    \item \nameref{Functions}
% \end{itemize}

\tableofcontents

\newpage
\addcontentsline{toc}{part}{Arrays}
\part*{\centering Arrays}

\noindent An array is a data structure consisting of a collection of values (integers, doubles, chars), of same memory size, each identified by an array index. \\ 



\begin{table}[H]
\makebox[\linewidth][c]
{
    \begin{tabular}{| c | c | c | c | c | c | c | c | c |} 
    \hline
    \textbf{Indexes} & 0 & 1 & 2 & 3 & 4 & 5 & 6 & 7 \\
    \hline
    \textbf{Values} & 70 & 84 & 65 & 90 & 60 & 82 & 76 & 91 \\
    \hline
    \end{tabular}
}
\caption*{This array holds marks of 8 students.}
\end{table}

\begin{minted}[bgcolor=bg, framesep=2mm]{cpp}
	
int marks[8] = {70, 84, 65, 90, 60, 82, 76, 91};

for (int i = 0; i < 8; i++)
{
    printf("%d\t", marks[i]);
}
printf("\n");

\end{minted}

\vspace{1cm}

\begin{itemize}
    \item In C and C++, arrays store data in \textbf{contiguous} memory.
    \item In C and C++, array indexes start from 0 instead of 1. This is called 0-based indexing.
    
    \begin{itemize}
        \item This means that for an array that hold 10 values, the valid indices range from 0 to 9, and \textbf{not} 1 to 10, or 0 to 10.
        
        \begin{itemize}
            \item In other words, for an array of size $n$, the last valid index is $n - 1$.
        \end{itemize}

        \item Some programming languages, like Python, have 1-based indexing.
    \end{itemize}

    \item The plural of ``index'' can be either indexes or indices.

\end{itemize}











\newpage
\addcontentsline{toc}{section}{Array Declaration \& Access}
\section*{Array Declaration \& Access}

\begin{minted}[bgcolor=bg, framesep=2mm]{cpp}
	
dataType arrayName[arraySize];

\end{minted}

% \vspace{1cm}

\begin{minted}[bgcolor=bg, framesep=2mm]{cpp}
	
int marks[8] = {70, 84, 65, 90, 60, 82, 76, 91};

double prices[4] = {9.99, 14.99, 49.99, 99.99};

\end{minted}

\vspace{1cm}

\noindent Once declared, the size and type of an array cannot be changed. \\

\noindent We can access individual elements of an array by specifying their index. \\

\begin{minted}[bgcolor=bg, framesep=2mm]{cpp}

#include <stdio.h>

int main()
{
    double prices[4] = {9.99, 14.9, 49.9, 99.9};

    printf("%lf\n", prices[0]);
    printf("%lf\n", prices[1]);
    printf("%lf\n", prices[2]);
    printf("%lf\n", prices[3]);

    scanf("%lf", &prices[2]);
    printf("%lf\n", prices[2]);

    return 0;
}

\end{minted}

\vspace{1cm}

\noindent As with any other variable, we need \textcolor{red}{\texttt{\&}} in \textcolor{red}{\texttt{scanf}} for array elements as well: 

\begin{center}
    \textcolor{red}{\texttt{scanf("\%lf", \&prices[2]);}}
\end{center}

\noindent \textbf{Reminder:} We need to use \textcolor{red}{\texttt{\%lf}} instead of \textcolor{red}{\texttt{\%f}} when dealing with \textcolor{red}{\texttt{double}} type. \\

% \newpage
\noindent We can also iterate through an array in a loop.

\begin{minted}[bgcolor=bg, framesep=2mm]{cpp}

int main()
{
    int marks[8] = {70, 84, 65, 90, 60, 82,
                    76, 91};

    for (int i = 0; i < 8; i++)
    {
        printf("%d\t", marks[i]);
    }
    printf("\n");

    return 0;
}

\end{minted}









\newpage
\addcontentsline{toc}{section}{Passing Arrays to Functions}
\section*{Passing Arrays to Functions}

\noindent There are two ways to pass arrays to functions:

\begin{minted}[bgcolor=bg, framesep=2mm]{cpp}

double calculateAverage(int* arr, int length)

\end{minted}

\begin{center}
    Or
\end{center}

\begin{minted}[bgcolor=bg, framesep=2mm]{cpp}

double calculateAverage(int arr[], int length)

\end{minted}

\vspace{1cm}

\noindent They both mean \href{https://stackoverflow.com/questions/6567742/passing-an-array-as-an-argument-to-a-function-in-c}{exactly the same thing}. You can use either. I personally prefer the former, that is, \textcolor{red}{\texttt{(int* arr, int length)}}. \\

\noindent When passing arrays to functions, generally, we should always pass the length of the array as well.

\begin{itemize}
    \item An array itself does not hold any information about its length. 
    \item Functions that operate on arrays need to know array length to determine how many times to iterate over (run loops for).
    \item For reasons we will see later, \textcolor{red}{\texttt{char}} arrays are exempted from this.
\end{itemize}

\newpage
\noindent Let us see a function that calculates the average from an array.  

\begin{align*}
    \text{Average} &= \frac{\sum x_i}{n} \\
\end{align*}

\begin{minted}[bgcolor=bg, framesep=2mm]{cpp}

#include <stdio.h>

double calculateAverage(int* arr, int length)
{
    int sum = 0;
    for (int i = 0; i < length; i++)
    {
        sum = sum + arr[i];
    }

    double average = sum / length;

    return average;
}

int main()
{
    int marks[8] = {70, 84, 65, 90, 60, 82, 
                    76, 91};

    double average = calculateAverage(marks, 8);
    printf("%lf\n", average);

    return 0;
}

\end{minted}

% \vspace{1cm}
\newpage
\noindent Can you write a function that calculates variance or standard deviation?

\begin{align*}
    \sigma^2 &= \frac{\sum x_i ^2}{n} - \left(\frac{\sum x_i}{n} \right)^2 & \sigma &= \sqrt{\frac{\sum x_i ^2}{n} - \left(\frac{\sum x_i}{n} \right)^2}\\
\end{align*}














\newpage
\addcontentsline{toc}{section}{Memory Stomping}
\section*{Memory Stomping}

\begin{minted}[bgcolor=bg, framesep=2mm]{cpp}
#include <stdio.h>

int main()
{
	int arr[2] = {0, 1};
	int b = 2;
	
	printf("arr[2] = %d\n", arr[2]);
	
	arr[2] = 10;    //Memory stomping!
	printf("\narr[2] = %d\n", arr[2]);
	
	printf("\nb = %d\n", b);
	
	return 0;
}
\end{minted}

























\newpage
\addcontentsline{toc}{section}{Two Dimensional Arrays}
\section*{Two Dimensional Arrays}

A two dimesional array is an array of arrays. Let us try to visualize an example:

\begin{table}[H]
\makebox[\linewidth][c]
{
    \begin{tabular}{| c | c | c | c |} 
    \hline
    &  &  &  \\
    & Column 0 & Column 1 & Column 2 \\
    &  &  &  \\
    \hline
    &  &  &  \\
    Row 0 & array[0, 0] = 1 & array[0, 1] = 5 & array[0, 2] = 9 \\
    &  &  &  \\
    \hline
    &  &  &  \\
    Row 1 & array[1, 0] = 5 & array[1, 1] = 3 & array[1, 2] = 5 \\
    &  &  &  \\
    \hline
    &  &  &  \\
    Row 2 & array[2, 0] = 7 & array[2, 1] = 2 & array[2, 2] = 3 \\
    &  &  &  \\
    \hline
    \end{tabular}
}
\end{table}

\noindent This 2d array or `array of arrays', represents a 3x3 matrix: 

\begin{align*}
    \begin{bmatrix}
        1 & 5 & 9 \\
        5 & 3 & 5 \\
        7 & 2 & 3 \\
    \end{bmatrix}
\end{align*}

\noindent Let us see how we can represent this in code (on next page):

\newpage

\begin{minted}[bgcolor=bg, framesep=2mm]{cpp}

#include <stdio.h>

int main()
{
    int array[3][3] = {{1, 5, 9}, 
                       {5, 3, 5}, 
                       {7, 2, 3}};

    for (int i = 0; i < 3; i++)
    {
        for (int j = 0; j < 3; j++)
        {
            printf("%d\t", array[i][j]);
        }
        printf("\n");
    }

    return 0;
}

\end{minted}

\vspace{1cm}

\noindent Here, \textcolor{red}{\texttt{\{1, 5, 9\}}}, \textcolor{red}{\texttt{\{5, 3, 5\}}} and \textcolor{red}{\texttt{\{7, 2, 3\}}} are three individual arrays within an array. \\

\noindent Generally, we need to use nested loops to iterate over two dimensional arrays. \\

\noindent \textbf{Note:} It is important to understand that real memory is linear, and a multidimensioanl array is actually always stored like a one dimensional array. \\



\newpage
\addcontentsline{toc}{section}{Passing Two Dimensional Arrays To Functions}
\section*{Passing Two Dimensional Arrays To Functions}

\noindent Further reading: \href{https://www.reddit.com/r/C_Programming/comments/t61f7o/question_passing_multidimensional_arrays_to/}{Question: Passing multidimensional arrays to functions.} \\

\noindent We have not yet covered memory allocation, and it is important to understand that unless a two dimensional array or a multidimensional array is allocated on heap memory, we cannot 
use the following syntax for passing such an array to a function: \\

\noindent \textcolor{red}{\texttt{printArray(int** array, int rows, int cols)}} \\

\noindent In other words, if we declared our array on stack (the way we have been doing until now), that is, \textcolor{red}{\texttt{int array[3][3]}}, we cannot use the above syntax to pass it to a function. \\

\noindent For the sake of completeness, the follwoing code demonstrates allocating a two dimensional array on heap and passing it to a function (next page).\\  

\newpage

\begin{minted}[bgcolor=bg, framesep=2mm]{cpp}
#include <stdio.h>

void printArray(int** arr, int rows, int cols)
{
    for (int i = 0; i < rows; i++)
    {
        for (int j = 0; j < cols; j++)
        {
            printf("%d ", arr[i][j]);
        }
        printf("\n");
    }
}

int main()
{
    int rows = 5, cols = 5;
    int **array;
    // Allocate rows
    array = (int **) malloc(rows * sizeof(int *));
    for (int i = 0; i < rows; i++)
    {
        // Allocate columns for each row
        array[i] = (int *) malloc(cols * sizeof(int));
    }

    printArray(rows, cols, array);

    // Free allocated memory
    for (int i = 0; i < rows; i++)
    {
        free(array[i]);
    }
    free(array);
    
    return 0;
}
\end{minted}

\newpage

\noindent For a two dimensional array declared on stack, this is how we can pass it to a function: \textcolor{red}{\texttt{void printArray(int array[9][9])}}. \\

\begin{minted}[bgcolor=bg, framesep=2mm]{cpp}
#include <stdio.h>

void printArray(int array[3][3])
{
    for (int i = 0; i < 3; i++)
    {
        for (int j = 0; j < 3; j++)
        {
            printf("%d ", array[i][j]);
        }
        printf("\n");
    }
}

int main()
{
    int array[3][3] = {{1, 5, 9}, 
                       {5, 3, 5}, 
                       {7, 2, 3}};

    for (int i = 0; i < 3; i++)
    {
        for (int j = 0; j < 3; j++)
        {
            printf("%d\t", array[i][j]);
        }
        printf("\n");
    }

	printf("\n");
	printArray(array);

    return 0;
}
\end{minted}

\newpage

\newpage
\addcontentsline{toc}{section}{Sudoku}
\section*{Sudoku}



\newpage
\noindent Given the following code, can you write a program to check if a Sudoku configuration is valid or invalid?

\begin{minted}[bgcolor=bg, framesep=2mm]{cpp}
#include <stdio.h>
#include <stdbool.h>

bool isSudokuRowValid(int matrix[9][9], int rowNo);
bool isSudokuColValid(int matrix[9][9], int colNo);
bool isSudokuSubMatrixValid(int matrix[9][9], int matrixNo);
bool isSudokuValid(int matrix[9][9]);
void printSudoku(int matrix[9][9]);

int main()
{
    int matrix[9][9] = { {9, 3, 0, 0, 7, 0, 0, 0, 0},
                         {6, 0, 0, 1, 9, 5, 0, 0, 0},
                         {0, 5, 8, 0, 0, 0, 0, 6, 0},
                         {8, 0, 0, 0, 6, 0, 0, 0, 3},
                         {4, 0, 0, 8, 0, 3, 0, 0, 1},
                         {7, 0, 0, 0, 2, 0, 0, 0, 6},
                         {0, 6, 0, 0, 0, 0, 2, 8, 0},
                         {0, 0, 0, 4, 1, 9, 0, 0, 5},
                         {0, 0, 0, 0, 8, 0, 0, 7, 9} };

    //int matrix[9][9] = { {9, 3, 0, 0, 7, 0, 0, 0, 0},
    //                        {6, 0, 0, 1, 9, 5, 0, 0, 0},
    //                        {0, 5, 8, 0, 0, 0, 0, 6, 0},
    //                        {8, 0, 0, 0, 6, 0, 0, 0, 3},
    //                        {4, 0, 0, 8, 0, 3, 0, 0, 1},
    //                        {7, 0, 0, 0, 2, 0, 0, 0, 6},
    //                        {0, 6, 0, 0, 0, 0, 2, 8, 0},
    //                        {0, 0, 0, 4, 1, 9, 7, 0, 5},
    //                        {0, 0, 0, 0, 8, 0, 0, 7, 9} };

    printSudoku(matrix);
    isSudokuValid(matrix);

    return 0;
}
\end{minted}










\newpage
\addcontentsline{toc}{part}{Strings}
\part*{\centering Strings}

\noindent A string is a \textcolor{red}{\texttt{char}} array of fixed size that is always null terminated. \\  

\noindent Consider the following string: 

\begin{minted}[bgcolor=bg, framesep=2mm]{cpp}

    char name[] = "Asif";

\end{minted}

\noindent This \textcolor{red}{\texttt{char}} array has \textbf{5} (not 4) characters: `A', `s', `i', `f' and `\textbackslash 0'.  \\

\noindent This string can contain 5 \textcolor{red}{\texttt{chars}}, has a length of 4, and contains 5 \textcolor{red}{\texttt{chars}}; the last \textcolor{red}{\texttt{char}}, \textcolor{red}{\texttt{`\textbackslash 0'}}, 
the null character, is used to determine the end of the string. \\

\noindent Before talking about what is null character, let us consider another similar string:

\begin{minted}[bgcolor=bg, framesep=2mm]{cpp}
    
    char name[50] = "Asif";

\end{minted}

\noindent This string can contain \textbf{50} \textcolor{red}{\texttt{chars}}, has a length of 4, and again contains 5 \textcolor{red}{\texttt{chars}}; as before, the last \textcolor{red}{\texttt{char}}, 
\textcolor{red}{\texttt{`\textbackslash 0'}}, the null character, is used to determine the end of the string. \\

\addcontentsline{toc}{section}{Null Character}
\section*{Null Character}

\noindent The null character is a control character with the value zero and is represented as a \textcolor{red}{\texttt{char}} by \textcolor{red}{\texttt{`\textbackslash 0'}}. \\

\noindent Recall that in C and C++, an array itself does not contain information about its size. That is why when we pass, say, integer or double array to a function, we also have to pass the 
length of the array as well. \\ 

\noindent For our convenience, however, we use the null character at the end of \textcolor{red}{\texttt{char}} strings. This lets us pass strings to functions like 
\textcolor{red}{\texttt{printf()}} without specifying their length. When such a function operates on a string, it keeps iterating until it encounters the null character.   

\begin{minted}[bgcolor=bg, framesep=2mm]{cpp}

    char name[50] = "Asif";
    printf("%s\n", name);

\end{minted}

\addcontentsline{toc}{section}{Initializing Strings}
\section*{Initializing Strings}

\begin{minted}[bgcolor=bg, framesep=2mm]{cpp}

    char c[] = "Asif";

    char c[50] = "Asif";

    char c[] = {'A', 's', 'i', 'f', '\0'};

    char c[5] = {'A', 's', 'i', 'f', '\0'};

\end{minted}

\vspace{1cm}
\noindent When using double quotes to initialize a string, we do not have to explicitly append the null character; the compiler does the job for us. \\



\newpage
\addcontentsline{toc}{section}{Scanning \& Printing Strings}
\section*{Scanning \& Printing Strings}

When scanning and printing strings, we use the \textcolor{red}{\texttt{\%s}} format specifier.

\begin{minted}[bgcolor=bg, framesep=2mm]{cpp}

#include <stdio.h>

int main()
{
    char str[100];

    printf("Enter string: ");
    scanf(" %s", str);
    printf("%s\n", str);

    return 0;
}

\end{minted}

% \newpage
\addcontentsline{toc}{section}{Accessing Individual Characters of Strings}
\section*{Accessing Individual Characters of Strings}

We access individual characters of strings or \textcolor{red}{\texttt{char}} arrays just like other arrays; by using square brackets \textcolor{red}{\texttt{[]}}.   

% newpage
\addcontentsline{toc}{section}{String Manipulation}
\section*{String Manipulation}

\noindent String manipulation refers to various operations on null-terminated strings, like finding the length of a string, copying one string into another string, string concatenation, etc. \\

\noindent The \textcolor{red}{\texttt{<string.h>}} header provides us with built in functions for string manipulation but we will write and use our own functions for now for the sake of practice. \\ 

\addcontentsline{toc}{subsection}{Finding Length of a String}
\subsection*{Finding Length of a String}

\begin{minted}[bgcolor=bg, framesep=2mm]{cpp}
#include <stdio.h>

int getStringLength(char* str)
{
	int length = 0;
	int index = 0;
	
	//while (str[index++] != '\0')
	while (str[index] != '\0')
	{
		index++;
		length++;
	}
	
	return length;
}

int main()
{
	char name[50] = "Asif";
	
	printf("%s\n", name);
	
	char str[100];
	
	printf("Enter string: ");
	scanf(" %s", str);
	printf("%s\n", str);

	printf("Length = %d\n", getStringLength(str));

	return 0;
}

\end{minted}

\newpage
\addcontentsline{toc}{section}{ASCII Table}
\section*{ASCII Table}

\begin{table}[H]
\makebox[\linewidth][c]
{
    \begin{tabular}{| c | c | c | c | c | c | c | c | c |} 
    \hline
    \textbf{Dec} & \textbf{Hex} & \textbf{Char} & \textbf{Dec} & \textbf{Hex} & \textbf{Char} & \textbf{Dec} & \textbf{Hex} & \textbf{Char} \\
    \hline
    0 & 00 & [NULL] & 32 & 20 & [SPACE] & 64 & 40 & @ \\
    1 & 01 & [SOH] & 33 & 21 & ! & 65 & 41 & A \\
    2 & 02 & [STX] & 34 & 22 & " & 66 & 42 & B \\
    3 & 03 & [ETX] & 35 & 23 & \# & 67 & 43 & C \\
    4 & 04 & [EOT] & 36 & 24 & \$ & 68 & 44 & D \\
    5 & 05 & [ENQ] & 37 & 25 & \% & 69 & 45 & E \\
    6 & 06 & [ACK] & 38 & 26 & \& & 70 & 46 & F \\
    7 & 07 & [BEL] & 39 & 27 & ' & 71 & 47 & G \\
    8 & 08 & [BS] & 40 & 28 & ( & 72 & 48 & H \\
    9 & 09 & [TAB] & 41 & 29 & ) & 73 & 49 & I \\
    10 & 0A & [LF] & 42 & 2A & * & 74 & 4A & J \\
    11 & 0B & [VT] & 43 & 2B & + & 75 & 4B & K \\
    12 & 0C & [FF] & 44 & 2C & , & 76 & 4C & L \\
    13 & 0D & [CR] & 45 & 2D & - & 77 & 4D & M \\
    14 & 0E & [SO] & 46 & 2E & . & 78 & 4E & N \\
    15 & 0F & [SI] & 47 & 2F & / & 79 & 4F & O \\
    16 & 10 & [DLE] & 48 & 30 & 0 & 80 & 50 & P \\
    17 & 11 & [DC1] & 49 & 31 & 1 & 81 & 51 & Q \\
    18 & 12 & [DC2] & 50 & 32 & 2 & 82 & 52 & R \\
    19 & 13 & [DC3] & 51 & 33 & 3 & 83 & 53 & S \\
    20 & 14 & [DC4] & 52 & 34 & 4 & 84 & 54 & T \\
    21 & 15 & [NAK] & 53 & 35 & 5 & 85 & 55 & U \\
    22 & 16 & [SYN] & 54 & 36 & 6 & 86 & 56 & V \\
    23 & 17 & [ETB] & 55 & 37 & 7 & 87 & 57 & W \\
    24 & 18 & [CAN] & 56 & 38 & 8 & 88 & 58 & X \\
    25 & 19 & [EM] & 57 & 39 & 9 & 89 & 59 & Y \\
    26 & 1A & [SUB] & 58 & 3A & : & 90 & 5A & Z \\
    27 & 1B & [ESC] & 59 & 3B & ; & 91 & 5B & [ \\
    28 & 1C & [FS] & 60 & 3C & < & 92 & 5C & \textbackslash \\
    29 & 1D & [GS] & 61 & 3D & = & 93 & 5D & ] \\
    30 & 1E & [RS] & 62 & 3E & > & 94 & 5E & \textasciicircum \\
    31 & 1F & [US] & 63 & 3F & ? & 95 & 5F & \_ \\
    \hline
    \end{tabular}

}
% \caption{Complete ASCII Character Table (0-127)}
% \label{tab:ascii_complete}
\end{table}

\begin{table}[H]
\makebox[\linewidth][c]
{
    \begin{tabular}{| c | c | c |} 
    \hline
    \textbf{Dec} & \textbf{Hex} & \textbf{Char} \\
    \hline
    96 & 60 & \textasciigrave \\
    97 & 61 & a \\
    98 & 62 & b \\
    99 & 63 & c \\
    100 & 64 & d \\
    101 & 65 & e \\
    102 & 66 & f \\
    103 & 67 & g \\
    104 & 68 & h \\
    105 & 69 & i \\
    106 & 6A & j \\
    107 & 6B & k \\
    108 & 6C & l \\
    109 & 6D & m \\
    110 & 6E & n \\
    111 & 6F & o \\
    112 & 70 & p \\
    113 & 71 & q \\
    114 & 72 & r \\
    115 & 73 & s \\
    116 & 74 & t \\
    117 & 75 & u \\
    118 & 76 & v \\
    119 & 77 & w \\
    120 & 78 & x \\
    121 & 79 & y \\
    122 & 7A & z \\
    123 & 7B & \{ \\
    124 & 7C & | \\
    125 & 7D & \} \\
    126 & 7E & \textasciitilde \\
    127 & 7F & [DEL] \\
    \hline
    \end{tabular}
}
\caption{ASCII Characters 96-127 (Vertical Layout)}
\end{table}

\newpage

\begin{table}[H]
\makebox[\linewidth][c]
{
    \begin{tabular}{| c | c | c | c | c | c | c | c | c | c | c | c |} 
    \hline
    \textbf{Dec} & \textbf{Hex} & \textbf{Char} & \textbf{Dec} & \textbf{Hex} & \textbf{Char} & \textbf{Dec} & \textbf{Hex} & \textbf{Char} & \textbf{Dec} & \textbf{Hex} & \textbf{Char} \\
    \hline
    0 & 00 & [NULL] & 32 & 20 & [SPACE] & 64 & 40 & @ & 96 & 60 & \textasciigrave \\
    1 & 01 & [SOH] & 33 & 21 & ! & 65 & 41 & A & 97 & 61 & a \\
    2 & 02 & [STX] & 34 & 22 & " & 66 & 42 & B & 98 & 62 & b \\
    3 & 03 & [ETX] & 35 & 23 & \# & 67 & 43 & C & 99 & 63 & c \\
    4 & 04 & [EOT] & 36 & 24 & \$ & 68 & 44 & D & 100 & 64 & d \\
    5 & 05 & [ENQ] & 37 & 25 & \% & 69 & 45 & E & 101 & 65 & e \\
    6 & 06 & [ACK] & 38 & 26 & \& & 70 & 46 & F & 102 & 66 & f \\
    7 & 07 & [BEL] & 39 & 27 & ' & 71 & 47 & G & 103 & 67 & g \\
    8 & 08 & [BS] & 40 & 28 & ( & 72 & 48 & H & 104 & 68 & h \\
    9 & 09 & [TAB] & 41 & 29 & ) & 73 & 49 & I & 105 & 69 & i \\
    10 & 0A & [LF] & 42 & 2A & * & 74 & 4A & J & 106 & 6A & j \\
    11 & 0B & [VT] & 43 & 2B & + & 75 & 4B & K & 107 & 6B & k \\
    12 & 0C & [FF] & 44 & 2C & , & 76 & 4C & L & 108 & 6C & l \\
    13 & 0D & [CR] & 45 & 2D & - & 77 & 4D & M & 109 & 6D & m \\
    14 & 0E & [SO] & 46 & 2E & . & 78 & 4E & N & 110 & 6E & n \\
    15 & 0F & [SI] & 47 & 2F & / & 79 & 4F & O & 111 & 6F & o \\
    16 & 10 & [DLE] & 48 & 30 & 0 & 80 & 50 & P & 112 & 70 & p \\
    17 & 11 & [DC1] & 49 & 31 & 1 & 81 & 51 & Q & 113 & 71 & q \\
    18 & 12 & [DC2] & 50 & 32 & 2 & 82 & 52 & R & 114 & 72 & r \\
    19 & 13 & [DC3] & 51 & 33 & 3 & 83 & 53 & S & 115 & 73 & s \\
    20 & 14 & [DC4] & 52 & 34 & 4 & 84 & 54 & T & 116 & 74 & t \\
    21 & 15 & [NAK] & 53 & 35 & 5 & 85 & 55 & U & 117 & 75 & u \\
    22 & 16 & [SYN] & 54 & 36 & 6 & 86 & 56 & V & 118 & 76 & v \\
    23 & 17 & [ETB] & 55 & 37 & 7 & 87 & 57 & W & 119 & 77 & w \\
    24 & 18 & [CAN] & 56 & 38 & 8 & 88 & 58 & X & 120 & 78 & x \\
    25 & 19 & [EM] & 57 & 39 & 9 & 89 & 59 & Y & 121 & 79 & y \\
    26 & 1A & [SUB] & 58 & 3A & : & 90 & 5A & Z & 122 & 7A & z \\
    27 & 1B & [ESC] & 59 & 3B & ; & 91 & 5B & [ & 123 & 7B & \{ \\
    28 & 1C & [FS] & 60 & 3C & < & 92 & 5C & \textbackslash & 124 & 7C & | \\
    29 & 1D & [GS] & 61 & 3D & = & 93 & 5D & ] & 125 & 7D & \} \\
    30 & 1E & [RS] & 62 & 3E & > & 94 & 5E & \textasciicircum & 126 & 7E & \textasciitilde \\
    31 & 1F & [US] & 63 & 3F & ? & 95 & 5F & \_ & 127 & 7F & [DEL] \\
    \hline
    \end{tabular}
}
% \caption{Complete ASCII Character Table (0-127)}
% \label{tab:ascii_complete}
\end{table}











\newpage
\addcontentsline{toc}{part}{Lab Tasks}
\part*{\centering Lab Tasks}

\addcontentsline{toc}{section}{Lab 09 Tasks}
\section*{Lab 09 Tasks}


\begin{enumerate}

\item Write a function \textcolor{red}{\texttt{void reverseArray (int* arr, int length)}} that reverses an array. Demonstrate by printing the array before and after reversing.

\item Write a function \textcolor{red}{\texttt{int findMax (int* arr, int length)}} that finds the maximum value in an array. Demonstrate this by writing a program that declares an array of 
5 variables, uses a \textcolor{red}{\texttt{for}} loop to take input from console (from user) using \textcolor{red}{\texttt{scanf}}, and prints the max by calling the function.

\item Write a function \textcolor{red}{\texttt{int findIndexOf (int* arr, int length, int value)}} that searches the array for \textcolor{red}{\texttt{value}} and returns the index. If 
\textcolor{red}{\texttt{value}} does not exist in array, the function should return $-1$.

\item \textbf{Student Grade Management}

A teacher wants to analyze class performance. Write a program that:

\begin{itemize}
    \item Stores marks of 50 students in an array (Use \textcolor{red}{\texttt{rand() \% (100 - 50) + 50}}).
    \item Calculates the class average (use a function).
    \item Finds how many students scored above 90\% (use a function).
\end{itemize}

Include \textcolor{red}{\texttt{<stdlib.h>}} in your code to use \textcolor{red}{\texttt{rand()}}.

\item Write a program to add two square matrices and display the result. 

\end{enumerate}


\newpage
\addcontentsline{toc}{section}{Lab 10 Tasks}
\section*{Lab 10 Tasks}

\begin{enumerate}
    \item Write a function \textcolor{red}{\texttt{void removeCommas(char* str)}} to remove all commas from a string. Do not forget to null-terminate the modified string. Use the following 
    string to test your code:

    \textcolor{red}{\texttt{char str[] = "C,o,,m,,,,,,i,c,,,,,  ,,,,, ,,,, s,a,n,s, r,e,i,g,n,,s s,u,p,r,e,,,,m,e,,,,";}}

    \item Write a function \textcolor{red}{\texttt{void copyString(char* destinationString, const char* sourceString)}} to copy one string into another. Make sure that 
    \textcolor{red}{\texttt{d1estinationString}} is at least as large as \textcolor{red}{\texttt{sourceString}}.

    \item Write a function to concatenate two strings.

\end{enumerate}






























\end{document}
